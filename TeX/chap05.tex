\chapter{Resultados e Discussões}

A Fig. \ref{clusterT1} apresenta à análise de agrupamentos das propriedades físicas analisadas por classes de rochas para o poço T$1$. Foi utilizado para tal foi adotado um padrão de cores para cada tipo litológico. 

\begin{figure}[H]
	\centering
	\setlength{\fboxsep}{8pt}
	\setlength{\fboxrule}{0.1pt}
	\fbox{
		\includegraphics[scale=0.5]{Imagens/cluterpocoT1.png}
	}
	\caption{Agrupamento de dados do poço T1.}
	\label{clusterT1}
\end{figure} 

Em vermelho escuro, se encontra o diabásio, a gradação de cores entre o vermelho claro e o amarelo, se encontra o embasamento, a gradação de cores entre o laranja e o verde claro encontra-se a dolomita, verde claro se encontra o folhelho $2$, a gradação de azul para azul escuro encontra-se o conglomerado, e a gradação que vai do amarelo ao azul são as subclasses de mistura de conglomerado com embasamento de $20\%$, $40\%$, $60\%$ e $80\%$, respectivamente.

É perceptível o notável contraste de variação de resistividade entre a rocha de origem ígnea, em contraste com as propriedades físicas das demais rochas de origem sedimentar e metamórfica. O agrupamento das rochas sedimentares formam um conjunto quase linear próximo a zero.

A Fig. \ref{clusterC1} apresenta à variação das propriedades físicas analisadas por agrupamento de classes de rochas para o poço C$1$.

\begin{figure}[H]
	\centering
	\setlength{\fboxsep}{8pt}
	\setlength{\fboxrule}{0.1pt}
	\fbox{
		\includegraphics[scale=0.5]{Imagens/cluterpocoC1.png}
	}
	\caption{Agrupamento de dados do poço C1.}
	\label{clusterC1}
\end{figure} 

Neste caso, o agrupamento das classes de rochas é mais evidente, no gráfico de raio-gama por densidade, que evidencia os $5$ litotipos distintamente. E, da mesma maneira, o gráfico de velocidade por densidade.


A Fig. \ref{clusterC2} apresenta à variação das propriedades físicas analisadas por agrupamento de classes de rochas para o poço C$2$. Em destaque, de vermelho, o litotipo diabásio. 

\begin{figure}[H]
	\centering
	\setlength{\fboxsep}{8pt}
	\setlength{\fboxrule}{0.1pt}
	\fbox{
		\includegraphics[scale=0.5]{Imagens/cluterpocoC2.png}
	}
	\caption{Agrupamento de dados do poço C2.}
	\label{clusterC2}
\end{figure} 

Na mesma forma, o agrupamento das classes de rochas é mais evidente, no gráfico de raio-gama por densidade, que evidencia os $5$ litotipos distintamente. E, da mesma maneira, o gráfico de velocidade por densidade.

\section{Treinamento}

A etapa de treinamento consiste em um ajuste de pesos dos neurônios da rede. Nesta fase, é identificado o neurônio que tem os valores dos pesos mais parecidos com os parâmetros de entrada da rede.  Por conseguinte, os diversos mapas são obtidos através dos sucessivos ciclos de treinamento ao longo do tempo. A Fig. \ref{SOM}(a), representa a organização da rede com apenas um ciclo de treinamento. Nesta imagem, a rede ainda não é capaz de identificar nenhuma litologia. Ao se aumentar o número de ciclos é perceptível que o ajuste dos pesos cria um conjunto de neurônios vencedores capazes de identificar as classes litológicas. Na quinta iteração, Fig. \ref{SOM}(b), as classes folhelho 2 e dolomita, por exemplo (cores mais azuis) ocupam a maior área do mapa. Já na milésima iteração, Fig. \ref{SOM}(d), a área azul é reduzida dando lugar as cores amarela e verde, que representam as subclasses de conglomerado e embasamento.

\begin{figure}[H]
\centering
\subfigure[ref1][Iteração 1]{\includegraphics[width=7.0cm]{Imagens/SOM1_2d.pdf}}
\qquad
\subfigure[ref2][Iteração 5]{\includegraphics[width=7.0cm]{Imagens/SOM5_2d.pdf}}
\qquad
\subfigure[ref3][Iteração 100]{\includegraphics[width=7.0cm]{Imagens/SOM100_2d.pdf}}
\qquad
\subfigure[ref4][Iteração 1000]{\includegraphics[width=7.0cm]{Imagens/SOM1000_2d.pdf}}
\qquad
\caption{Mapas auto-organizáveis e sua evolução temporal.}
\label{SOM}
\end{figure}

Os mapas da Fig. \ref{SOM} apresentam as zonas do hiperplano especializadas em identificar as classes de rochas. O código numérico $1$ representa folhelho, $2$ dolomita, $3$ diabásio, $4$ conglomerado, $5$ embasamento, $6$ mistura conglomerado/embasamento $80\%$, $7$ mistura conglomerado/embasamento $60\%$, $8$ mistura conglomerado/embasamento $40\%$ e $9$ mistura conglomerado/embasamento $20\%$ . A Tab. \ref{codigos} faz um paralelo entre o código numérico utilizado com \textit{output} da rede e as litologias do modelo.

\begin{table}[H]
	\centering
	\begin{tabular}{c|c}
		
		Litologia                    & Código numérico \\ % Note a separação de col. e a quebra de linhas
		\hline                                                             % para uma linha horizontal
		Folhelho 2                 &  1\\
		Dolomita  		            &  2 \\
		Diabásio    	            &  3 \\
		Conglomerado          &  4 \\
		Conglomerado 80\% &  5  \\
		Conglomerado 60\%&  6 \\
		Conglomerado 40\%&  7\\
		Conglomerado 20\%&  8 \\
		Embasamento          &  9 \\
	 % não é preciso quebrar a última linha
		
	\end{tabular}
	\label{codigos}
	\caption{Tabela de referência para conversão do padrão numérico em litologia.}
\end{table}

A Fig. \ref{convergencia} apresenta o teste de convergência da rede neuronal.

\begin{figure}[H]
	\centering
	\setlength{\fboxsep}{8pt}
	\setlength{\fboxrule}{0.1pt}
	\fbox{
		\includegraphics[scale=0.3]{Imagens/conv070917.png}
	}
	\caption{Teste de convergência da rede.}
	\label{convergencia}
\end{figure} 

O teste de convergência é realizado durante a fase de treinamento e mostra que a rede se encontra estabilizada em  $1000$ ciclos de treinamento com $28$ erros de classificação, ou seja, um erro de $4\%$. Isto significa ser inócuo aumentar a iteração afim de diminuir o erro. 



\section{Identificação}

A seguir são apresentados os resultados da etapa de classificação da rede foram acrescentados os resultados dos classificadores euclideanos e de Mahalanobis. Nesta fase, dois poços foram utilizados chamados de poços C$1$ e C$2$. O primeiro destes localizado a SW do perfil, Fig. \ref{modelo}, possui $7$km de profundidade. A saída da rede, para o poço C$1$ está localizada ao lado direito da Fig. \ref{Class C1}. Ao lado esquerdo é apresentada o poço original. Abaixo é mostrado uma breve estatística deste processo de identificação da rede. Ao lado esquerdo do poço identificado pela rede estão os resultados obtidos pelos classificadores euclideano e de Mahalanobis. 


\begin{figure}[H]
	\centering
	\setlength{\fboxsep}{8pt}
	\setlength{\fboxrule}{0.1pt}
	\fbox{
		\includegraphics[scale=0.5]{Imagens/IDC1020118.png}
	}
	\caption{Dado de saída da rede para o poço de classificação C1.}
	\label{Class C1}
\end{figure} 


O processo de identificação foi repetido, contudo, desta vez, para o caso do ao poço C$2$. Este localiza-se mais a NE do perfil, Fig. \ref{modelo}, no topo de um alto estrutural com igual profundidade de $7$ km. A saída da rede, para o poço C$2$ está localizada ao lado direito da Fig. \ref{Class C2}. Ao lado esquerdo é apresentada o poço original. Abaixo é mostrado uma breve estatística do processo de identificação.  Ao lado esquerdo do poço identificado pela rede estão os resultados obtidos pelos classificadores euclideano e de Mahalanobis.  


\begin{figure}[H]
	\centering
	\setlength{\fboxsep}{8pt}
	\setlength{\fboxrule}{0.1pt}
	\fbox{
		\includegraphics[scale=0.5]{Imagens/IDC2020118.png}
	}
	\caption{Dado de saída da rede para o poço de classificação C2.}
	\label{Class C2}
\end{figure} 


Em ambos os casos de identificação, o número de neurônios vitoriosos igualou-se ao total de neurônios da rede, Tab. \ref{Estatistica da rede}. Isto indica o máximo de aproveitamento durante os processos, com um tempo de máquina atingindo $25$ segundos. 


\begin{table}[H]
	\centering
	\caption{}
	\label{Estatistica da rede}
	\begin{tabular}{@{}lcc@{}}
		\toprule
		\multicolumn{3}{c}{Estatística da Rede}         \\ \midrule
		Dados                       & Poço C1 & Poço C2 \\
		Dados de treinamento        & 697     & 697     \\
		Dados a serem classificados & 699     & 698     \\
		Neurônios da Rede           & 400     & 400     \\
		Neurônios vitoriosos        & 400     & 400     \\
		Neurônios sem uso           & 0       & 0       \\
		Erro                        & 11      & 5       \\ \bottomrule
	\end{tabular}
\end{table} 


 Os classificadores apresentaram dois comportamentos distintos. No poço C$1$, o classificador de Euclides apresentou $42$ erros confundindo rochas do embasamento com rochas com $20\%$ de conglomerado e embasamento. No poço C$2$, o classificador de Euclides apresentou um maior número de erros $12$ associados ao embasamento cristalino classificando alguns pontos como conglomerado com $20\%$ de conglomerado e embasamento. 

Em contra-partida, o classificador de Mahalanobis apresentou $79$ erros, no total do poço C$1$ trocando as rochas do embasamento e conglomerado por dois tipos específicos de rocha: as rochas do $20\%$ e $60\%$ CE. E apresentou $128$ erros bem distribuídos ao longo das rochas do embasamento e conglomerado, no poço C$2$ confundindo-as com rochas de $60\%$ e $20\%$ de conglomerado com embasamento. 


\section{Dado Real: treinamento}

Foram performados ao todo $6$ testes no que tange o dado real. Os parâmetros testados da rede neuronal foram o número de neurônios e épocas\footnote{Época é definido pelo número de iterações ao longo do tempo.} Analogamente ao dado sintético foram analisadas os mapas auto-organizáveis e as curvas de convergência da rede. Para cada teste foram gerados três mapas SOMs que representam o início o meio e o fim do processo de treinamento e aprendizagem da rede. O por fim as curvas de convergência apontam todo o processo de aprendizado da rede. 

Como a taxa de amostragem do dado real é alta optou-se por aumentar a dimensão do hiperplano da rede. O teste $01$ foi conduzido em uma malha composta por $1600$ neurônios, $10$ épocas, $976$ neurônios vitoriosos e um custo computacional de $6,388$s em tempo de máquina. o conjunto de mapas são apresentados na Fig. \ref{SOMt01}.

\begin{figure}[H]
	\centering
	\subfigure[ref1][Início]{\includegraphics[width=7.0cm]{Imagens/SOM1t01.pdf}}
	\qquad
	\subfigure[ref2][Metade]{\includegraphics[width=7.0cm]{Imagens/SOM2t01.pdf}}
	\qquad
	\subfigure[ref3][Final]{\includegraphics[width=7.0cm]{Imagens/SOM3t01.pdf}}
	\qquad
	\caption{Mapas auto-organizáveis e sua evolução temporal. A figura (a) mostra a rede com $40$X$40$ neurônios no início do processo de treinamento. A figura (b) apresenta a rede no meio do processo de treinamento e (c) a rede no final do processo de treinamento.}
	\label{SOMt01}
\end{figure}

A convergência do teste $01$ assumiu um comportamento exponencial até a décima época. A Fig \ref{Conv01}. 

\begin{figure}[H]
	\centering
	\setlength{\fboxsep}{8pt}
	\setlength{\fboxrule}{0.1pt}
	\fbox{
		\includegraphics[scale=0.3]{Imagens/conv01.png}
	}
	\caption{Convergência do teste $01$. Inicialmente a curva aproxima-se de uma função exponencial.}
	\label{Conv01}
\end{figure} 


O teste $02$ foi conduzido em uma malha composta por $1600$ neurônios, $100$ épocas, $1600$ neurônios vitoriosos e um custo computacional de $62,01$s em tempo de máquina. o conjunto de mapas apresentados na Fig. \ref{SOMt02}.

\begin{figure}[H]
	\centering
	\subfigure[ref1][Início]{\includegraphics[width=7.0cm]{Imagens/SOM1t02.pdf}}
	\qquad
	\subfigure[ref2][Metade]{\includegraphics[width=7.0cm]{Imagens/SOM3t02.pdf}}
	\qquad
	\subfigure[ref3][Final]{\includegraphics[width=7.0cm]{Imagens/SOM2t02.pdf}}
	\qquad
	\caption{Mapas auto-organizáveis e sua evolução temporal. A figura (a) mostra a rede com $40$X$40$ neurônios no início do processo de treinamento. A figura (b) apresenta a rede no meio do processo de treinamento e (c) a rede no final do processo de treinamento.}
	\label{SOMt02}
\end{figure}

A convergência do teste $02$ assume um comportamento de queda indicando que a rede ainda aprende. A Fig \ref{Conv02}. 

\begin{figure}[H]
	\centering
	\setlength{\fboxsep}{8pt}
	\setlength{\fboxrule}{0.1pt}
	\fbox{
		\includegraphics[scale=0.3]{Imagens/conv02.png}
	}
	\caption{Convergência do teste $02$.}
	\label{Conv02}
\end{figure} 

O teste $03$ foi conduzido em uma malha composta por $1600$ neurônios, $1000$ épocas, $1600$ neurônios vitoriosos e um custo computacional de $610,12$s em tempo de máquina. o conjunto de mapas apresentados na Fig. \ref{SOMt03}.

\begin{figure}[H]
	\centering
	\subfigure[ref1][Início]{\includegraphics[width=7.0cm]{Imagens/SOM1t03.pdf}}
	\qquad
	\subfigure[ref2][Metade]{\includegraphics[width=7.0cm]{Imagens/SOM3t03.pdf}}
	\qquad
	\subfigure[ref3][Final]{\includegraphics[width=7.0cm]{Imagens/SOM2t03.pdf}}
	\qquad
	\caption{Mapas auto-organizáveis e sua evolução temporal. A figura (a) mostra a rede com $40$X$40$ neurônios no início do processo de treinamento. A figura (b) apresenta a rede no meio do processo de treinamento e (c) a rede no final do processo de treinamento.}
	\label{SOMt03}
\end{figure}

A convergência do teste $03$ indica queda próximo a época $1000$. A Fig \ref{Conv03}. 


\begin{figure}[H]
	\centering
	\setlength{\fboxsep}{8pt}
	\setlength{\fboxrule}{0.1pt}
	\fbox{
		\includegraphics[scale=0.3]{Imagens/conv03.png}
	}
	\caption{Convergência do teste $03$.}
	\label{Conv03}
\end{figure} 

O teste $04$ foi conduzido em uma malha composta por $400$ neurônios, $10$ épocas, $400$ neurônios vitoriosos e um custo computacional de $1,45$s em tempo de máquina. o conjunto de mapas apresentados na Fig. \ref{SOMt04}.

\begin{figure}[H]
	\centering
	\subfigure[ref1][Início]{\includegraphics[width=7.0cm]{Imagens/SOM1t04.pdf}}
	\qquad
	\subfigure[ref2][Metade]{\includegraphics[width=7.0cm]{Imagens/SOM3t04.pdf}}
	\qquad
	\subfigure[ref3][Final]{\includegraphics[width=7.0cm]{Imagens/SOM2t04.pdf}}
	\qquad
	\caption{Mapas auto-organizáveis e sua evolução temporal. A figura (a) mostra a rede com $20$X$20$ neurônios no início do processo de treinamento. A figura (b) apresenta a rede no meio do processo de treinamento e (c) a rede no final do processo de treinamento.}
	\label{SOMt04}
\end{figure}

A convergência do teste $04$ indica queda próximo a época $10$. A Fig \ref{Conv04}. 

\begin{figure}[H]
	\centering
	\setlength{\fboxsep}{8pt}
	\setlength{\fboxrule}{0.1pt}
	\fbox{
		\includegraphics[scale=0.3]{Imagens/conv04.png}
	}
	\caption{Convergência do teste $04$.}
	\label{Conv04}
\end{figure} 

O teste $05$ foi conduzido em uma malha composta por $400$ neurônios, $100$ épocas, $400$ neurônios vitoriosos e um custo computacional de $15,60$s em tempo de máquina. o conjunto de mapas apresentados na Fig. \ref{SOMt05}.


\begin{figure}[H]
	\centering
	\subfigure[ref1][Início]{\includegraphics[width=7.0cm]{Imagens/SOM1t04.pdf}}
	\qquad
	\subfigure[ref2][Metade]{\includegraphics[width=7.0cm]{Imagens/SOM3t04.pdf}}
	\qquad
	\subfigure[ref3][Final]{\includegraphics[width=7.0cm]{Imagens/SOM2t04.pdf}}
	\qquad
	\caption{Mapas auto-organizáveis e sua evolução temporal. A figura (a) mostra a rede com $20$X$20$ neurônios no início do processo de treinamento. A figura (b) apresenta a rede no meio do processo de treinamento e (c) a rede no final do processo de treinamento.}
	\label{SOMt05}
\end{figure}

A convergência do teste $05$ indica queda próximo a época $100$. A Fig \ref{Conv05}. 

\begin{figure}[H]
	\centering
	\setlength{\fboxsep}{8pt}
	\setlength{\fboxrule}{0.1pt}
	\fbox{
		\includegraphics[scale=0.3]{Imagens/conv05.png}
	}
	\caption{Convergência do teste $05$.}
	\label{Conv05}
\end{figure} 


O teste $06$ foi conduzido em uma malha composta por $400$ neurônios, $1000$ épocas, $400$ neurônios vitoriosos e um custo computacional de $153,82$s em tempo de máquina. o conjunto de mapas apresentados na Fig. \ref{SOMt05}.


\begin{figure}[H]
	\centering
	\subfigure[ref1][Início]{\includegraphics[width=7.0cm]{Imagens/SOM1t04.pdf}}
	\qquad
	\subfigure[ref2][Metade]{\includegraphics[width=7.0cm]{Imagens/SOM3t04.pdf}}
	\qquad
	\subfigure[ref3][Final]{\includegraphics[width=7.0cm]{Imagens/SOM2t04.pdf}}
	\qquad
	\caption{Mapas auto-organizáveis e sua evolução temporal. A figura (a) mostra a rede com $20$X$20$ neurônios no início do processo de treinamento. A figura (b) apresenta a rede no meio do processo de treinamento e (c) a rede no final do processo de treinamento.}
	\label{SOMt06}
\end{figure}

A convergência do teste $06$ indica queda próximo a época $1000$. A Fig \ref{Conv06}. 

\begin{figure}[H]
	\centering
	\setlength{\fboxsep}{8pt}
	\setlength{\fboxrule}{0.1pt}
	\fbox{
		\includegraphics[scale=0.3]{Imagens/conv06.png}
	}
	\caption{Convergência do teste $06$.}
	\label{Conv06}
\end{figure} 

\section{Dado Real: identificação}

A seguir são apresentados os resultados de classificação da rede para os seis testes realizados. As rochas presentes nos poços analisados são identificadas através de códigos numéricos de acordo com a Tab. \ref{codigosreal}. 

\begin{table}[H]
	\centering
	\begin{tabular}{c|c}
		
		Litologia                    & Código numérico \\ % Note a separação de col. e a quebra de linhas
		\hline                                                             % para uma linha horizontal
		Calcilutito                   &  6              \\
		Calcarenito  		          &  8              \\
		Diabásio    	              &  65              \\
		Conglomerado                  &  42              \\
		Diamictito                    &  44              \\
		Arenito                       &  49              \\
		Siltito                       &  54              \\
		Folhelho                      &  57              \\
		Meta-siltito                  &  76              \\
		Calcário Cristalino           &  2              \\
		% não é preciso quebrar a última linha
		
	\end{tabular}
	\label{codigosreal}
	\caption{Tabela de referência para conversão do padrão numérico em litologia.}
\end{table}

A Fig. \ref{IDt01} apresenta os resultados de identificação da rede para o teste $01$.

\begin{figure}[H]
	\centering
	\includegraphics[scale=0.4]{Imagens/result01.png}
	\caption{Identificação do teste 01. São apresentados dois gráficos. A esquerda encontra-se o resultado de classificação da rede e a direita o dado do poço original.}
	\label{IDt01}
\end{figure} 

A Tab. \ref{Estatistica do teste $01$} mostra o resumo da identificação realizada no teste $01$.

\begin{table}[H]
	\centering
	\caption{}
	\label{Estatistica do teste $01$}
	\begin{tabular}{@{}lc@{}}
		\toprule
		\multicolumn{2}{c}{Estatística do Teste $01$}         \\ \midrule
		Parâmetros                  & 1BN0002SC \\
		Época                       & 10       \\
		Erros                       & 2229       \\
		Neurônios da Rede           & 1600       \\
		Neurônios vitoriosos        & 976       \\
		Neurônios sem uso           & 624         \\
		Tempo de máquina            & 6,388 s   \\ \bottomrule
	\end{tabular}
\end{table} 

A Fig. \ref{IDt02} apresenta os resultados de identificação da rede para o teste $02$.

\begin{figure}[H]
	\centering
		\includegraphics[scale=0.4]{Imagens/result02.png}
	\caption{Identificação do teste 02. São apresentados dois gráficos. A esquerda encontra-se o resultado de classificação da rede e a direita o dado do poço original.}
	\label{IDt02}
\end{figure} 

A Tab. \ref{Estatistica do teste $02$} mostra o resumo da identificação realizada no teste $02$.

\begin{table}[H]
	\centering
	\caption{}
	\label{Estatistica do teste $02$}
	\begin{tabular}{@{}lc@{}}
		\toprule
		\multicolumn{2}{c}{Estatística do Teste $02$}         \\ \midrule
		Parâmetros                  & 1BN0002SC \\
		Época                       & 100       \\
		Erros                       & 2242       \\
		Neurônios da Rede           & 1600       \\
		Neurônios vitoriosos        & 1600       \\
		Neurônios sem uso           & 0         \\
		Tempo de máquina            & 62,01 s   \\ \bottomrule
	\end{tabular}
\end{table} 

A Fig. \ref{IDt03} apresenta os resultados de identificação da rede para o teste $03$.

\begin{figure}[H]
	\centering
	\includegraphics[scale=0.4]{Imagens/result03.png}
	\caption{Identificação do teste 03. São apresentados dois gráficos. A esquerda encontra-se o resultado de classificação da rede e a direita o dado do poço original.}
	\label{IDt03}
\end{figure} 


A Tab. \ref{Estatistica do teste $03$} mostra o resumo da identificação realizada no teste $03$.

\begin{table}[H]
	\centering
	\caption{}
	\label{Estatistica do teste $03$}
	\begin{tabular}{@{}lc@{}}
		\toprule
		\multicolumn{2}{c}{Estatística do Teste $03$}         \\ \midrule
		Parâmetros                  & 1BN0002SC \\
		Época                       & 1000       \\
		Erros                       & 2226       \\
		Neurônios da Rede           & 1600       \\
		Neurônios vitoriosos        & 1600       \\
		Neurônios sem uso           & 0         \\
		Tempo de máquina            & 610,12 s   \\ \bottomrule
	\end{tabular}
\end{table} 

A Fig. \ref{IDt04} apresenta os resultados de identificação da rede para o teste $04$.

\begin{figure}[H]
	\centering
	\includegraphics[scale=0.4]{Imagens/result04.png}
	\caption{Identificação do teste 04. São apresentados dois gráficos. A esquerda encontra-se o resultado de classificação da rede e a direita o dado do poço original.}
	\label{IDt04}
\end{figure} 

A Tab. \ref{Estatistica do teste $04$} mostra o resumo da identificação realizada no teste $04$.

\begin{table}[H]
	\centering
	\caption{}
	\label{Estatistica do teste $04$}
	\begin{tabular}{@{}lc@{}}
		\toprule
		\multicolumn{2}{c}{Estatística do Teste $04$}         \\ \midrule
		Parâmetros                  & 1BN0002SC \\
		Época                       & 10       \\
		Erros                       & 2219       \\
		Neurônios da Rede           & 400       \\
		Neurônios vitoriosos        & 400       \\
		Neurônios sem uso           & 0         \\
		Tempo de máquina            & 1,45 s   \\ \bottomrule
	\end{tabular}
\end{table} 

A Fig. \ref{IDt05} apresenta os resultados de identificação da rede para o teste $05$.

\begin{figure}[H]
	\centering
	\includegraphics[scale=0.4]{Imagens/result05.png}
	\caption{Identificação do teste 05. São apresentados dois gráficos. A esquerda encontra-se o resultado de classificação da rede e a direita o dado do poço original.}
	\label{IDt05}
\end{figure} 

A Tab. \ref{Estatistica do teste $05$} mostra o resumo da identificação realizada no teste $05$.

\begin{table}[H]
	\centering
	\caption{}
	\label{Estatistica do teste $05$}
	\begin{tabular}{@{}lc@{}}
		\toprule
		\multicolumn{2}{c}{Estatística do Teste $05$}         \\ \midrule
		Parâmetros                  & 1BN0002SC \\
		Época                       & 100       \\
		Erros                       & 2408       \\
		Neurônios da Rede           & 400       \\
		Neurônios vitoriosos        & 400       \\
		Neurônios sem uso           & 0         \\
		Tempo de máquina            & 15,60 s   \\ \bottomrule
	\end{tabular}
\end{table} 


A Fig. \ref{IDt06} apresenta os resultados de identificação da rede para o teste $06$.

\begin{figure}[H]
	\centering
	\includegraphics[scale=0.4]{Imagens/result01.png}
	\caption{Identificação do teste 06. São apresentados dois gráficos. A esquerda encontra-se o resultado de classificação da rede e a direita o dado do poço original.}
	\label{IDt06}
\end{figure} 

A Tab. \ref{Estatistica do teste $06$} mostra o resumo da identificação realizada no teste $06$.

\begin{table}[H]
	\centering
	\caption{}
	\label{Estatistica do teste $06$}
	\begin{tabular}{@{}lc@{}}
		\toprule
		\multicolumn{2}{c}{Estatística do Teste $06$}         \\ \midrule
		Parâmetros                  & 1BN0002SC \\
		Época                       & 1000       \\
		Erros                       & 2240       \\
		Neurônios da Rede           & 400       \\
		Neurônios vitoriosos        & 400       \\
		Neurônios sem uso           & 0         \\
		Tempo de máquina            & 153,82 s   \\ \bottomrule
	\end{tabular}
\end{table} 

