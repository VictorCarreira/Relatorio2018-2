\begin{abstract}


O campo do aprendizado de máquina aborda a criação de programas computacionais que tenham a capacidade de automaticamente melhorarem a si próprios com o passar do tempo. Técnicas de classificação tais como as métricas Euclideanas e de Mahalanobis são considerados técnicas clássicas de aprendizado de máquina. Tais medidas de similaridades são referidas na literatura como medidas de distância. O classificador Euclideano inclui o cálculo de um centróide no espaço de atributos, enquanto que o classificador de mahalanobis leva em consideração a forma do espaço de atributos.  Já um mapa auto-organizado (SOM) é inspirado no córtex cerebral. Uma SOM é um algoritmo baseado em um grafo orientado cujos vértices são unidades fundamentais chamadas de neurônios artificiais e o que governa as interações entre os neurônios são os pesos e vizinhança. Esse neurônios artificiais mudam o seu peso a medida que as iterações vão ocorrendo. Rochas sedimentares refletem o ambiente sedimentar no qual ela fora formada. Essas rochas em especial possuem assinaturas específicas no que tange as propriedades físicas da matéria registradas em dados de perfilagem de poços. Este relatório apresenta resultados de simulações em dados sintéticos que representam a hipótese de um poço perfurado em uma Bacia de Sinéclise semelhante  a Bacia Sedimentar do Paraná que é o alvo de estudo deste projeto.  Para a escolha do melhor classificador levou-se em consideração um exame de três técnicas diferentes de aprendizado de máquina o classificador euclideano, o classificador  de mahalanobis e a SOM. Esse estudo foi realizado com base em dados sintéticos controlados que apontaram um melhor desempenho de classificação para a rede neuronal com apenas $11$ erros de classificação para o poço C$1$ e $5$ erros de classificação para o poço C$2$.  Na atual fase do projeto foram aplicados dados reais que indicaram que o conjunto de propriedades analisadas não foram eficazes para o treinamento e identificação da rede que apresentou erros de classificação litológica no valor de $2219$ erros de $3008$ dados para o poço 1BN0002SC.  

\end{abstract}

