\chapter{Conclusões}

O teste de convergência da rede, Fig \ref{convergencia}, realizado durante a etapa de treinamento, indicou que o número de erros não iria diminuir após o milésimo ciclo de treinamento. Sendo o resultado, Fig. \ref{SOM}, deste teste usado como parâmetro para o número de repetições realizadas para os casos de identificação da rede. 

Os diagramas de velocidades por densidade e o de velocidade por raio-gama, Fig. \ref{clusterT1}, Fig. \ref{clusterC1} e Fig. \ref{clusterC2}, apresentaram os agrupamentos mais bem separados. Portanto estas propriedades físicas (densidade, velocidade e raio-gama) tem uma importância relativa maior ,na classificação das litologias dos poços C$1$ e C$2$. 

A saída da rede aponta que o maior caso de erros ocorreram em uma única classe de rocha, a do embasamento. Esses erros fizeram com que conglomerados fossem classificados como rochas do embasamento, nos dois casos dos poços de classificação, o poço C$1$ e o poço C$2$.  Uma das razões pode ser o fato das misturas de conglomerado e embasamento serem finas demais para a rede conseguir realizar uma identificação de padrão. Ou pelo fato dos conjuntos de propriedades físicas da mistura de $20\%$ se aproximar das propriedades físicas que representam o litotipo embasamento. 

O menor número de erros relativos encontrados, no poço C$2$, Fig. \ref{Class C2}, deve-se a escolha da alocação do furo, no perfil. O poço C$2$ localiza-se em um baixo estrutural, atingindo menos de $1$km do embasamento. Entretanto, o poço C$1$, Fig. \ref{Class C1}, encontra-se em um alto estrutural, divergindo do poço C$2$ e produzindo, consequentemente, os maiores erros relativos encontrados.    

O classificador de Euclides apresentou mais erros do que a rede neuronal de Kohonen com $42$ erros para o poço C$1$ e $12$ erros para o poço C$2$. E o classificador de Mahalanobis apresentou o resultado de identificação de poços com os maiores erros $79$ e $128$ respectivamente para os poços C$1$ e C$2$. 

Tal desempenho dos classificadores se deu por conta da existência de uma falha normal aonde foi escolhida a alocação do furo, Fig. \ref{modelo}. Nesta situação simulada há uma mistura entre os clusters em todos os espaços bi-dimensionais de propriedades analisadas tanto para o conglomerado quanto para o embasamento cristalino, Fig. \ref{clusterT1}. Portanto a definição dos centroides dos agrupamentos de propriedades ficam longe das distribuições ideais preditas no teste analítico do capítulo \ref{introducao}, na seção \ref{teste}.  

No tocante ao dado real, foram realizados seis testes com o objetivo de observar o comportamento da rede neuronal. Os dados que compõe a função que descreve a litologia para a rede é composta pelas seguintes entradas: SP, RLAT, TTI, TOT. Confirme indicado nas figuras \ref{1BN0002SCa} e \ref{1BN0002SCb} respectivamente. Essas entradas foram escolhidas de acordo com a disponibilidade de dados presentes nos poços de treinamento e classificação. A escolha de ambos os poços obedeceram o critério de proximidade entre si. Dado a alta amostragem presentes em ambos o poços optou-se por aumentar o número de neurônios da rede nos primeiros testes.

Os mapas auto-organizáveis concernentes ao teste $01$ indicam que a rede não consegue aprender a identificar os tipos litológicos presentes no poço de treinamento. Esse indicativo é apontado pela maior área dedicada a identificação de somente uma rocha, especificamente o calcilutito (vide figura \ref{SOMt01}) em detrimento as demais rochas. Em contrapartida, a convergência indica que a rede está aprendendo conforme indicado na figura \ref{Conv01}. Esse comportamento é justificado pelo número de épocas empregadas no presente teste. É importante ressaltar que, conforme apresentado na tabela \ref{Estatistica do teste $01$}, apenas $976$ neurônios são utilizados na etapa de identificação da rede, indicando um subaproveitamento de neurônios. A fase de identificação da rede indicou um erro de $2229$ amostras conforme indicado na figura \ref{IDt01} e na tabela \ref{Estatistica do teste $01$}.

O teste $02$ indicou que a rede não identifica os tipos litológicos presentes no poço de treinamento, comportamento semelhante ao teste $01$. Esse indicativo é apontado pela maior área dedicada a identificação de somente uma rocha, especificamente o calcário cristalino (vide figura \ref{SOMt02}) em detrimento as demais rochas. Em contrapartida, a convergência indica que a rede está aprendendo conforme indicado na figura \ref{Conv02} diminuindo o número de erro durante a fase de treinamento. Esse comportamento é justificado pelo número de épocas empregadas no presente teste. A tabela \ref{Estatistica do teste $02$} apontou que todos os neurônios foram utilizados no teste. A fase de identificação da rede indicou um erro de $2242$ amostras conforme indicado na figura \ref{IDt02} e na tabela \ref{Estatistica do teste $02$}.

O teste $03$ indicou que a rede não identifica os tipos litológicos presentes no poço de treinamento, comportamento semelhante ao teste $01$ e $02$. Esse indicativo é apontado pela maior área dedicada a identificação de somente uma rocha, especificamente o calcário cristalino (vide figura \ref{SOMt03}) em detrimento as demais rochas mostrando um comportamento muito semelhante ao teste $02$. Em contrapartida, a convergência indica que a rede está aprendendo conforme indicado na figura \ref{Conv03} diminuindo o número de erro durante a fase de treinamento. Esse comportamento é justificado pelo número de épocas empregadas no presente teste. A tabela \ref{Estatistica do teste $03$} apontou que todos os neurônios foram utilizados no teste. A fase de identificação da rede indicou um erro de $2226$ amostras conforme indicado na figura \ref{IDt03} e na tabela \ref{Estatistica do teste $03$}.

Conforme indícios de subaproveitamento de neurônios da redes presentes no teste $01$ no quarto teste optou-se por utilizar uma rede menor. Neste caso em particular o teste $04$ apresentou $100\%$ de aproveitamento dos neurônios. Contudo, o comportamento na fase de treinamento permaneceu semelhante aos testes anteriores. Esse indicativo é apontado pela maior área dedicada a identificação de somente uma rocha, especificamente o calcário cristalino (vide figura \ref{SOMt04}) em detrimento as demais rochas mostrando nos teste anteriores. Em contrapartida, a convergência indica que a rede está aprendendo conforme indicado na figura \ref{Conv04} diminuindo o número de erro durante a fase de treinamento. Esse comportamento é justificado pelo número de épocas empregadas no presente teste. A tabela \ref{Estatistica do teste $04$} apontou que todos os neurônios foram utilizados no teste. A fase de identificação da rede indicou um erro de $2219$ amostras conforme indicado na figura \ref{IDt04} e na tabela \ref{Estatistica do teste $04$}.

O teste $05$ apresentou comportamento na fase de treinamento semelhante aos testes anteriores. Esse indicativo é apontado pela maior área dedicada a identificação de somente uma rocha, especificamente o calcário cristalino (vide figura \ref{SOMt05}) em detrimento as demais rochas mostrando nos teste anteriores. Em contrapartida, a convergência indica que a rede está aprendendo conforme indicado na figura \ref{Conv05} diminuindo o número de erro durante a fase de treinamento. Esse comportamento é justificado pelo número de épocas empregadas no presente teste. A tabela \ref{Estatistica do teste $05$} apontou que todos os neurônios foram utilizados no teste. A fase de identificação da rede indicou um erro de $2408$ amostras conforme indicado na figura \ref{IDt05} e na tabela \ref{Estatistica do teste $05$}.

O teste $06$ apresentou comportamento na fase de treinamento semelhante aos testes anteriores. Esse indicativo é apontado pela maior área dedicada a identificação de somente uma rocha, especificamente o calcário cristalino (vide figura \ref{SOMt06}) em detrimento as demais rochas mostrando nos teste anteriores. Em contrapartida, a convergência indica que a rede está aprendendo conforme indicado na figura \ref{Conv06} diminuindo o número de erro durante a fase de treinamento. Esse comportamento é justificado pelo número de épocas empregadas no presente teste. A tabela \ref{Estatistica do teste $06$} apontou que todos os neurônios foram utilizados no teste. A fase de identificação da rede indicou um erro de $2240$ amostras conforme indicado na figura \ref{IDt06} e na tabela \ref{Estatistica do teste $06$}.

Na média os erros na fase de identificação da rede giraram em torno de $2230$, aproximadamente $75\%$ classificações erradas. Os testes apontaram que tal desempenho pode ser justificado principalmente pela escolha das entradas da rede. Tais propriedades físicas não definem com eficácia rochas. Futuramente será necessário estabelecer uma base de dados de propriedades físicas e rochas que definam com clareza horizontes de rochas em poços.    


 